\section{Future work}
\label{sec:futurework}



%Our current algorithms assume that we know the degree distribution of the graph.  This is fine if we have the graph of a real-world social network and we are trying to build a model to compare it to.  However, it fails if we are given the motif counts only.

%Fortunately, social networks tend to have power-law degree distributions, which means their distributions can be described by a single parameter $\alpha$, where $\alpha$ is the magnitude of the exponent.  (We also need a normalization constant, which can be found from the number of edges.)  For the final paper we will build a model to predict $\alpha$ from the motif counts.  We think we can do this because graphs with a lot of high-edge motifs should be denser, so the degree distribution should have a heavier tail.

%Models to try include linear regression, neural networks, and stochastic gradient descent.
