\begin{abstract}

%Networks with similar motif distributions tend to have similar functions in the real world~\cite{chuanqi}.

In this paper, we present a heuristic method to generate graphs with specific motif counts.  Motif counts imply many structural properties of a graph, such as clustering coefficient and degree distribution (Section~\ref{sec:alpha}), and can be used to cluster networks into meaningful real-world categories~\cite{chuanqi}.  By constructing graphs with specific motif counts, we hope to build models that will imitate the functionality of specific networks.

Our method is based on hill-climbing.  We perform successive transformations on an initial graph, keeping the ones that reduce error and discarding the ones that don't.  Running this algorithm on $9$ real-world networks shows substantial improvement over the baseline.  On our metabolic network, our model's motif counts were identical to the motif counts of the original network.  On our power grid network, the average relative error between the motif counts of our model and the network was $0.006$.  All networks saw at least a 58\% decrease in average relative error, with most networks seeing much better performance.

%Social motif has widely used for clustering large social network. However, graph generation algorithms often focus on degree distribution and lack of motif distribution. In this paper, we define the motif-driven graph generation problem and present a heuristic method which focus constructing the approximate motif distribution of large network. Our experiments on 67 real social network motif network distribution show that our approach obtains the generated graph with similar motif distribution within relevant error less then 0.02 in 10 hours.

\end{abstract} 